\documentclass[10pt,twocolumn,letter]{article}
\usepackage{styles/usenix-style}
\usepackage{styles/ka-style}
\usepackage{xspace,ifthen,graphicx,listings}
\usepackage{styles/ka-style}

\usepackage[utf8]{inputenc}
\usepackage[inline]{enumitem}
\usepackage{parskip} % disable indentation for new paragraphs, increased margin-bottom instead
\usepackage[english]{babel}
\usepackage{csquotes}
\usepackage[style=alphabetic]{biblatex}
\addbibresource{literature.bib}

\usepackage[
   pdfauthor={Christian Schwarz},
   pdftitle={Seminar Report - Light Weight Contexts},
   pdfsubject={An OS Abstraction for Safety and Performance}, 
   pdfkeywords={}
]{hyperref}

\setlength{\marginparwidth}{2cm} % to make todonotes fit in twocolumn
\usepackage{todonotes}

\usepackage{blindtext}

\begin{document}
\title{%
  % document class article doesn't support subtitles, let's hack them
  {\normalfont \normalsize Seminar Report on}\\%
  Light Weight Contexts\\%
  {\normalfont \normalsize An OS Abstraction for Safety and Performance}\\%
  {\normalfont \small %
    James Litton\textsuperscript{1,2}
    Anjo Vahldiek-Oberwagner\textsuperscript{2}
    Eslam Elnikety\textsuperscript{2}
    Deepak Garg\textsuperscript{2}
    Bobby Bhattacharjee\textsuperscript{1}
    Peter Druschel\textsuperscript{2}
  }\\
  {\normalfont \small
    \textsuperscript{1}University of Maryland, College Park 
    \textsuperscript{2}Max Planck Institute for Software Systems
  }%
}
\author{Report by Christian Schwarz}
\date{2019}

\maketitle

\begin{abstract}
  \blindtext
\end{abstract}

\section{Introduction}\label{intro}

Application Compartmentalization as in seminar presentation. Ignore Session Isolation \& Snapshots?

Line of throught:

Modern app architecture commonly emphasizes modularization \& information hiding (Parnas) to achieve testability, maintainability, exchangability, reusability.
The extreme of reusability are large libraries used by multiple independent applications.

However, chasm between development time and runtime:
PLs enforce encapsulation at compile time, but the resulting program runs in a single protection domain, i.e. a process, at runtime:
shared address space with shared heap, shared file descriptor table, shared system-level privilege (user id, group id, capabilities(SYS\_CAP\_), system call) between all modules of the program.
Consequently, an exploitable vulnerability in a single module not only compromises that module but the entire application.

The generic solution: decomponsition of the application into smaller units that execute in separate protection domains.
Generic questions / problems / tradeoffs:
\begin{itemize}
  \item threat model
  \item definition of what makes a protection domain
  \item how to integrate the decomposition into different protection domains into the PL / runtime?
  \item applicability to existing PLs \& code bases
  \item how to maintain application performance
\end{itemize}

\subsection{Structure of this Report}
\blindtext


\section{Design}\label{design}
We find it most helpful to develop the general idea behind lwCs by starting from the status quo of the well-established abstraction of processes \& threads.
Conventionally, processes define an execution environment which is shared by one or more threads.
The execution environment consists of an address space, a file descriptor table and a representation of the process's system-wide privileges (\textit{credential}).
A thread assumes \textbf{two closely related roles}:
\textbf{first}, it represents a \textbf{single unit of control flow} within that environment.
Control flow has associated state like instruction pointer, stack pointer, general purpose register contents, FPU state, \dots.
That state resides in a CPU's registers while the thread is executing on a CPU, or in the thread control block (TCB) kernel data structure.
The \textbf{second} role of a thread is that of a \textbf{scheduling entity}:
The scheduler time-multiplexes threads onto CPU cores and implements the concept of blocking \& waiting between threads.
Parts of the necessary state for this task are tracked in the thread control block (TCB).
The canonical relationship between processes and threads as described above is visualized in figure XXXTODOXXX.

The authors propose \textit{light weight contexts} (lwCs) as a new OS abstraction that represents protection domain \& control flows within that protection domain.

Each thread within a process executes within a single lwC at a given time.
Multiple threads can execute simultaneously within an lwCs.
lwCs are represented by file descriptors and thus tangible from user-space.
Threads can switch protection domains by switching between lwCs.



\subsection{Usage Examples}\label{usage}
\subsection{Design Critique}\label{dsgn:crit}

\begin{description}
  \item[+] foo
  \item[--] bar
\end{description}

\section{Threat Model}\label{threat}

\section{Implementation \& Evaluation}\label{eval}

\subsection{Evaluation Critique}\label{eval:crit}

\section{Related Work}\label{rel}

\begin{itemize}
  \item Application Compartmentalization
  \begin{itemize}
    \item Principle of least Privilege
    
    \item Language-Technology-Based
    \begin{itemize}
      \item NaCL \& WASM
      \item BPF \& eBPF
      \item Software Fault Isolation
      \item Memory-safe Runtime (Java, CLR?) %working title, need proper PL word for it
    \end{itemize}
    
    \item OS-Based
      \begin{itemize}
        \item Capability Systems
        \begin{itemize}
          \item Fiasco-OC / seL4
          \item FreeBSD Capsicum
        \end{itemize}
        \item Process-Based Privsep
        \begin{itemize}
          \item Provos, OpenSSH
        \end{itemize}
        \item Hybrids (isolation at some level)
        \begin{itemize}
          \item wedge
          \item shreds (uses ARM memory domains)
          \item \textbf{light weight contexts}
        \end{itemize}
      \end{itemize}

    \item Hardware-Based
    \begin{itemize}
      \item CHERI
    \end{itemize}
  \end{itemize}
  
  \item Process Snapshotting \& Rollback
  \item Session Isolation in CGI applications?
\end{itemize}

\section{Conclusion}\label{conclusion}

% Cite all the literature, not just the one we referenced in the text.
\nocite{*}
\clearpage
\printbibliography

\end{document}
